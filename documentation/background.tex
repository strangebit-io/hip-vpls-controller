\chapter{Background}

In this section, we are going to describe some background material. 
We are going to start with the cryptographic primitives, such as 
symmetric key encryption/decryption algorithms, and then move on 
to the discussion of the Virtual Private LAN Services and what kind 
of problems they solve. We will then conclude the discussion in this 
chapter with the basic information on Host Identity Protocol as it 
is in the core of the solution which we are discussing in this 
document. To make the background material more or less complete, 
we are going to touch alternative on Secure Socket Layer (we 
will here mention why this protocol is not used in the core of our 
architecture and is only used for the control-plane communications 
between the HIP-Switches and the HIP-controller). With these final 
words, we are going to conclude the current chapter of this work. 

\section{Cryptography}
Cryptography forms the basis for secure telecommunications nowadays. 
SSL, TLS, SSH, Tacacs+, IPSec, DKIM, and DNSSec are only a few well-known 
telecommunication protocols that use cryptography to prevent such well-known 
attacks as eavesdropping, tampering, denial of message origin, etc. Modern 
cryptography is based on hard-core mathematics and non-trivial algorithms 
(such as random number generation, discrete logarithm problem, rings, fields, 
Euclidean algorithm, factorization of big numbers, etc.) 

\subsection{Symmetric cryptography}
Symmetric key cryptography is just perfect for data-plane traffic as it offers 
low processing times (when compared to asymmetric key cryptography). As the 
name implies, symmetric key cryptography uses the same secret key to encrypt 
and decrypt messages~\cite{Stinson:Cryptography}. On one hand it is the main reason why these algorithms 
are so fast. On the other hand, and this is the main limitation of the type 
of cryptography: symmetric keys are hard to distribute and revoke without using 
more sophisticated symmetric key schemes. 

As of today, several symmetric key cryptography algorithms, such as Advanced 
Encryption Standard (AES), Triple DES (3DES), and Twofish, offer advantageous 
processing speed and sufficient security levels~\cite{Stinson:Cryptography}. In our prototype implementation 
of HIP-VPLS we are using AES with 256 bits keys to perform encryption and 
decryption of data-plane traffic. Moreover, since NanoPI R2S - hardware that 
we employ to run our Software Defined Network (SDN) code - has support for 
on-chip instructions to boost the encryption and decryption of arbitrary long 
message blocks. In other words, we perform AES operations directly in the tiny 
computer's Central Processing Unit (CPU). We are going to devote a separate 
section on the implementation of the hardware accelerated AES encryption and 
decryption routines by the CPU.

\subsection{Asymmetric cryptography}

Asymmetric key cryptography, as the name suggests, uses two separate keys to 
encrypt and decrypt the messages. Since the encryption uses big number 
exponentiations (such as RSA) and multiplications (such as EDDSA), as well 
as modular arithmetics, the performance of these algorithms is considerably 
worse compared to symmetric cryptography algorithms~\cite{Stinson:Cryptography}. 

However, since one is allowed to expose the public part of the key to anyone, 
and since this key is only required to encrypt the message and only the person 
who holds the private part of the key (secret part of the key) can decrypt the 
message, efficient key distribution and revocation can be organized, at the 
cost of extra CPU cycles. Moreover, digital signature schemes can be implemented 
with no hassle by encrypting the message with the private key and then making 
decryption plausible only with a public key (exposed to everyone). Many distributed 
versions of signature/encryption also exist in the literature broadening the 
application landscape of this type of cryptography. In our work, we use RSA and 
ECDSA in HIP protocol to generate message signatures. The HIP protocol also uses 
public keys to derive permanent identifiers for the HIP switches.
 

\subsection{Hash functions}

Hash functions are one-way mathematical functions used to generate the so-called 
fingerprints of a message (MACs)~\cite{Stinson:Cryptography}. In other words, given arbitrary long input messages, 
a fixed size universally unique message (typically 128, 160, and 256 bits) is produced. 
Ideally, it should be computationally impossible to reverse the function to find the 
original message (or image) given the hash (or fingerprint). Hash functions are used 
in digital signatures to compress the message before signing it with public key 
cryptography algorithms. Modern hash functions should guarantee that no collisions 
are possible for the hash function (in other words, it should be hard to find two 
distinct images that will both produce the same hash code). 

Keyed versions of hash functions are also widely spread. For example, Hash MAC (HMAC) 
is used in a symmetric setting when both parties share the key. This type of algorithm 
is used for the authentication and identification process. HMACs are typically used 
to protect the data-plane traffic from the forgery attacks. 


\subsection{Key exchange protocols}

Key exchange algorithms are crucial to many security solutions. Key exchange algorithms 
allow parties to negotiate secret keys over insecure communication channels. Diffie-Hellman 
and the improved variant Elliptic Curve DH are well-known key exchange algorithm variants~\cite{Stinson:Cryptography}. 
We have implemented both variants of the algorithms. Note that, in general, key exchange 
algorithms need to be protected by signature algorithms to exclude Man-In-The-Middle attacks.

\section{Virtual Private LAN Service}

Virtual Private LAN Services (or VPLS) are pretty standard nowadays. Companies build VPLS to 
provide Layer-2 services for branch offices: VPLS are typically built as overlays on top of 
Layer-3 (IP). For example, when a frame arrives at VPLS provider equipment (PE), it is encapsulated 
into an IP packet and is sent out to all other VPLS network elements comprising emulated LAN. 
Security of such overlays is important for obvious reasons: customers do not want their corporate 
traffic to be sniffed and analyzed. In this work, we build such a secure overlay with Host Identity 
Protocol. We also introduce a specially crafted control-plane protocol to configure the nodes centrally. 
More on HIP-VPLS can be found here~\cite{hipvpls}.

\section{Host Identity Protocol}

Internet was designed initially so that the Internet Protocol (IP) address has a dual role: it 
is the locator, so that the routers can find the recipient of a message, and it is an identifier 
so that the upper layer protocols (such as TCP and UDP) can make bindings (for example, transport 
layer sockets use IP addresses and ports to make connections). This becomes a problem when a networked 
device roams from one network to another, and so the IP address changes, leading to failures in 
upper-layer connections. The other problem is the establishment of an authenticated channel between 
the communicating parties. In practice, when making connections, the long-term identities of the parties 
are not verified. Of course, solutions such as SSL can readily solve the problem at hand. However, SSL 
is suitable only for TCP connections, and most of the time, practical use cases include only secure web 
surfing and the establishment of VPN tunnels. Host Identity Protocol, on the other hand, is more flexible: 
it allows peers to create authenticated secure channels on the network layer, so all upper-layer protocols 
can benefit from such channels. More on the protocol can be found in~\cite{gurtov:hip}.

HIP relies on the 4-way handshake to establish an authenticated session. During the handshake, the 
peers authenticate each other using long-term public keys and derive session keys using Diffie-Hellman 
or Elliptic Curve (EC) Diffie-Hellman algorithms. To combat the denial-of-service attacks, HIP also 
introduces computational puzzles. 

HIP uses a truncated hash of the public key as an identifier in the form of an IPv6 address and 
exposes this identifier to the upper layer protocols so that applications can make regular 
connections (for example, applications can open regular TCP or UDP socket connections). At the 
same time, HIP uses regular IP addresses (both IPv4 and IPv6 are supported) for routing purposes. 
Thus, when the attachment of a host changes (and so does the IP address used for routing purposes), 
the identifier, which is exposed to the applications, stays the same. HIP uses a particular 
signaling routine to notify the corresponding peer about the locator change. More information 
about HIP can be found in RFC 7401. 

\section{Secure socket layer (SSL)}
Secure socket layer (SSL)~\cite{ssl} is an application layer solution to secure TCP connections. 
SSL was standardized in RFC 6101. And was designed to prevent eavesdropping, man 
in-the-middle attacks, tampering and message forgery. In SSL the communicating 
hosts can authenticate each other with help of longer term identities - public key certificates.
SSL is greate for building VPN tunnels and protecting upper layer protocols such as HTTP.